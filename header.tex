%
% Diplomarbeit mit LaTeX
% ===========================================================================
% This is part of the book "Diplomarbeit mit LaTeX".
% Copyright (c) 2002, 2003, 2005, 2007, 2008 Tobias Erbsland
% Copyright (c) 2005, 2006 Andreas Nitsch
% See the file main.tex for copying conditions.
%

%
% A. DOKUMENTKLASSE
% ---------------------------------------------------------------------------
%

%
%  1. Definieren der Dokumentklasse.
%     Wir verwenden die KOMA-Script Klasse 'scrbook' für ein Buch.
%
\documentclass[%
	pdftex,%              PDFTex verwenden da wir ausschliesslich ein PDF erzeugen.
	oneside,%             Einseitiger Druck.
	12pt,%                Grosse Schrift, besser geeignet für A4.
	parskip=half,%        Halbe Zeile Abstand zwischen Absätzen.
	headsepline,%         Linie nach Kopfzeile.
	footsepline,%         Linie vor Fusszeile.
	bibtotocnumbered,%    Literaturverzeichnis im Inhaltsverzeichnis nummeriert einfügen.
]{scrartcl}

\usepackage{abstract}

% Hurenkinder und Schusterjungen verhindern
% http://mrunix.de/forums/showthread.php?t=66436
%\clubpenalty10000
%\widowpenalty10000
%\displaywidowpenalty=10000

%
%  2. Festlegen der Zeichencodierung des Dokuments und des Zeichensatzes.
%     Wir verwenden 'Latin1' (ISO-8859-1) für das Dokument,
%     und die 'T1' codierung für die Schrift.
%
\usepackage[utf8]{inputenc}
\usepackage[T1]{fontenc}
%\usepackage{setspace}
%\usepackage[onehalfspacing]{setspace}

%
%  4. Paket für die Lokalisierung ins Deutsche laden.
%     Wir verwenden neue deutsche Rechtschreibung mit 'ngerman'.
%
\usepackage[ngerman]{babel}


%
%  5. Paket für Anführungszeichen laden.
%     Wir setzen den Stil auf 'swiss', und verwenden so die Schweizer Anführungszeichen.
%
%\usepackage[style=swiss]{csquotes}
\usepackage[babel,german=quotes]{csquotes}

%
%  7. Paket um Grafiken im Dokument einbetten zu können.
%     Im PDF sind GIF, PNG, und PDF Grafiken möglich.
%
\usepackage{graphicx}

%
%  8. Pakete für mathematischen Textsatz.
%
\usepackage{amsmath}
\usepackage{amssymb}
%%\usepackage{dsfont}
%%\usepackage{mathtools}

%
% 13 Schriftart
%
%\usepackage{goudysans}
\usepackage{lmodern}
%\usepackage{libertine}

%
%  5. Farbeinstellungen für die Links im PDF Dokument.
%
%\hypersetup{%
%	colorlinks=false,%        Aktivieren von farbigen Links im Dokument (keine Rahmen)
%	linkcolor=LinkColor,%    Farbe festlegen.
%	citecolor=LinkColor,%    Farbe festlegen.
%	filecolor=LinkColor,%    Farbe festlegen.
%	menucolor=LinkColor,%    Farbe festlegen.
%	urlcolor=LinkColor,%     Farbe von URL's im Dokument.
%	bookmarksnumbered=true%  Überschriftsnummerierung im PDF Inhalt anzeigen.
%}

%
%  6. Einstellungen für das 'listings' Paket.
%
\usepackage{color}
\definecolor{ListingBackground}{gray}{.9}
\usepackage[savemem]{listings}
\lstloadlanguages{TeX} % TeX sprache laden, notwendig wegen option 'savemem'
\lstset{%
	language=Java,				 % Sprache des Quellcodes
	numbers=left,            % Zelennummern links
	stepnumber=1,            % Jede Zeile nummerieren.
	numbersep=5pt,           % 5pt Abstand zum Quellcode
	numberstyle=\tiny,       % Zeichengrösse 'tiny' für die Nummern.
	breaklines=true,         % Zeilen umbrechen wenn notwendig.
	breakautoindent=true,    % Nach dem Zeilenumbruch Zeile einrücken.
	postbreak=\space,        % Bei Leerzeichen umbrechen.
	tabsize=2,               % Tabulatorgrösse 2
	basicstyle=\ttfamily\footnotesize, % Nichtproportionale Schrift, klein für den Quellcode
	showspaces=false,        % Leerzeichen nicht anzeigen.
	showstringspaces=false,  % Leerzeichen auch in Strings ('') nicht anzeigen.
	extendedchars=true,      % Alle Zeichen vom Latin1 Zeichensatz anzeigen.
	captionpos=b,            % sets the caption-position to bottom
	keywordstyle=\color{blue}, % Farbe für die Keywords wie public, void, object u.s.w.
	%commentstyle=\color{green}, % Farbe der Kommentare
	%stringstyle=\color{green}, % Farbe der Zeichenketten
	backgroundcolor=\color{ListingBackground}} % Hintergrundfarbe des Quellcodes setzen.

