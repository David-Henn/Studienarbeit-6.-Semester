\section*{Zusammenfassung}

Diese Studienarbeit entstand im Rahmen des IT-Projekts  \emph{KSM}. KSM steht für ,,Kybernetisches System Modell`` und ist eine Java Applikation die es ermöglicht komplexe Systeme und Prozesse zu modellieren und anschließend zu simulieren um dadurch Rückschlüsse auf das echte System in der Realität zu ziehen.

Ziel dieser Arbeit ist es, den Reifegrad der Software, sowie des Projekts an sich zu steigern. Außerdem soll in der Anzeige \emph{SumChart Diagramm} die Funktion der Bereichsmarkierung verbessert, und erweitert werden.

In diesem Zuge wurde ein umfassendes Bugfixing und Refactoring des Programms und der Oberfläche vorgenommen. Außerdem wurde die Bereichsmarkierung von einfachen Ellipsen, hin zu flexiblen Polygonen erweitert
\newpage
\section*{Abstract}

This work was created in the context of the IT-project \emph{KSM}. KSM stands for ,,Kybernetisches System Modell`` and is a Java application, which allows modelling and simulating complex systems and processes to make conclusions to the behaviour of the system in reality.

The objective of this work is to raise the level of maturity of the software and the project itself. Furthermore the area-marking functionality of the view \emph{SumChart} shall be improved and enhanced.

In this regard an extensive bugfixing and refactoring of the software and its user interface has been carried out. In addition, the area-marking functionality has been advanced from simple ellipses to flexible polygons.