\section{Erste Phase}

Wie in Kapitel \ref{planung} beschrieben, wurde in der ersten Phase des Projektes versucht den Zustand des Gesamtprojektes zu verbessern. Da dies im Allgemeinen viele kurze Aufgaben sind, sowie während dieser Phase ständig neue Aufgaben identifiziert werden, wurde mit Scrum ein äußerst agiles Vorgehensmodell gewählt, das es dem Team ermöglicht diese Aufgaben sehr flexibel, aber doch strukturiert abzuarbeiten.

\subsection{Das Scrum Modell}

In diesem Abschnitt soll kurz auf das Scrum Modell, sowie der Einsatz in diesem Projekt eingegangen werden.

Im Gegensatz zu traditionellen Entwicklungsmodellen geht Scrum iterativ vor. Einzelne Iterationen werden in Scrum ,,Sprint'' genannt. Vor jedem Sprint legt das Projektteam fest, welche Aufgaben in dieser Zeit abgearbeitet werden sollen. Diese Aufgaben werden im Sprint-Backlog festgehalten. Am Ende eines solchen Sprints muss eine lauffähige Version stehen. Da ein Sprint nur eine Laufzeit von zwei bis vier Wochen hat, kann es durchaus passieren, dass einzelne Aufgaben innerhalb eines Sprints nicht fertiggestellt werden können. Dies ist jedoch nicht weiter schlimm, solange die Version trotzdem lauffähig bleibt. Nicht fertiggestellte Aufgaben werden im nächsten Sprint  weitergeführt. 

Während eines Sprints stehen die Entwicklern in engem Kontakt zueinander. Jeden Tag gibt es ein kurzes Scrum Meeting, in dem jeder Entwickler kurz darlegt, welche Aufgaben er am Tag davor bearbeitet hat und was er an diesem Tag zu tun gedenkt. Dies stellt sicher, dass stets das gesamte Team weiß, wo sich das Projekt zu diesem Zeitpunkt befindet und was für Entscheidungen getroffen wurden. Der gesamte Scrum Prozess wird in \cite{bib:agil} genau beschrieben.

Im KSM Projekt wurde eine Sprintdauer von jeweils zwei Wochen festgelegt. Das Sprint-Backlog, sowie das übergeordnete Projekt-Backlog wurden in Form eines Excel-Do\-ku\-ments\footnote{Siehe Anhang} geführt. Die Aufgaben in den Backlogs wurden von den einzelnen Teammitgliedern definiert. Als Scrum Master, der den Prozess überwacht und koordiniert fungierte Herr Tobias Dreher, der durch das Thema seiner Studienarbeit den größten Überblick über die Baustellen des Programms besitzt.

\subsection{Ergebnis}

Während dieser Stabilisierungsphase wurde der Produktreifegrad des KSM Projekts deutlich verbessert. 
\begin{itemize}
\item Durch viele kleine und größere Änderungen wurden Inkonsistenzen im Design, sowie der Benutzerführung beseitigt
\item Die Icons des Programms wurden ausgetauscht und durch aussagekräftigere Symbole ersetzt
\item Es wurde ein Build Prozess definiert, der es ermöglicht, eine auslieferbare Version des Programms zu generieren
\item Es wurden viele Codeleichen entfernt
\item eine Projekthomepage wurde erstellt
\item Das Projektverzeichnis wurde strukturiert und vereinheitlicht
\end{itemize}

Durch diese und viele weitere Änderungen wurde die Benutzbarkeit des Programms gesteigert, sowie der Entwicklungsprozess geordnet und vereinheitlicht.



