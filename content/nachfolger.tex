\section{Nachfolger}

Dieses Kapitel richtet sich an alle, die die Arbeit am Projekt KSM weiterführen. Es soll ein kurzer Überblick über die auch nach dieser Studienarbeit vorhandenen, oder neu hinzugekommenen Baustellen gegeben werden. 

Das KSM Projekt befindet sich zur Zeit in einer interessanten Phase. Das Programm an sich ist mittlerweile 10 Jahre alt. Viele Studenten haben sich jeweils für ein Jahr mit dem Quellcode befasst und so sieht er auch aus. Es wurden viele Funktionen implementiert und neue Erkenntnisse gewonnen. Teilweise fußen die Funktionen auf Java Klassen, die mittlerweile als obsolet eingestuft wurden. Es wurde versucht, so viele dieser Klassen wie möglich zu entfernen und durch aktuellere Implementierungen der Funktionalitäten zu ersetzen, doch manche Klassen sind einfach zu tief im Programm verwurzelt. Ebenso ist der Quellcode über die Jahre immer komplexer und unübersichtlicher geworden. Am Anfang eingeführte Architekturen sind über die Jahre vergessen worden und somit nicht mehr existent. Eine Modularisierung ist so gut wie unmöglich. Im Rahmen der Studienarbeiten des diesjährigen Projektteams wurde zwar versucht, den Code zu entflechten und mit einer Plugin-Architektur eine gute Möglichkeit entwickelt, das Chaos zu ordnen, doch für diese Aufgabe, die quasi einer Neuimplementierung gleich kommt, fehlte einfach die Zeit und der Wille von Herr Schubert für ein Jahr auf neue Funktionalitäten zu verzichten. Neuimplementierung ist trotzdem das Stichwort. Eine der Studienarbeiten dieses Jahres beschäftigte sich intensiv mit dem Eclipse Framework, das durch den OSGI Unterbau eine sehr gute Möglichkeit bietet ein Programm durch eine modulare Struktur übersichtlich zu halten. Alle neuen Aktivitäten sollten sich darauf konzentrieren, das alte KSM Programm als Plugin für die Eclipse Oberfläche neu zu implementieren. Das bisherige Programm sollte als ein Prototyp gesehen werden, der gezeigt hat, dass die Idee, Systeme durch einen Graphen zu modellieren sehr gut und leicht verständlich ist. Mit den während der Implementierung gewonnenen Erkenntnissen sollte nun ein strukturierter Neuanfang auf Basis des Eclipse Frameworks gestartet werden. Auch hierfür wurde bereits der erste Grundstein gelegt, indem die Funktionalität implementiert wurde, die Systeme, die in den alten XML Strukturen gespeichert wurden in das neue, besser strukturierte und mittels XSD definierte XML Format zu übertragen.

Ein weiterer interessanter Entwicklungszweig des KSM Projekts ist das QKSM. Das QKSM Projekt beschäftigt sich mit der Simulation des Systems auf Basis qualitativer Parameter, wie zum Beispiel ,,gut``, ,,viel`` oder ,,niedrig``. Das QKSM Projekt kam aufgrund vieler Abstimmungsschwierigkeiten mit den Projektverantwortlichen in diesem Jahr leider nicht so weit, wie es mit den beiden Studenten hätte kommen können. Nun sollte dieser Stand ebenfalls eingefroren werden und als Plugin für die neue Eclipse Version neu implementiert werden.

Sollte es sich allerdings nicht umgehen lassen, auch an der alten Version des KSM Programms arbeiten zu müssen, so sollte ab jetzt auf Refactoring verzichtet werden. Diese Zeit ist besser in die Neuimplementierung investiert. Mögliche Weiterentwicklungspunkte ergeben sich zum Beispiel durch die Funktionalitäten, die in dieser Studienarbeit implementiert wurden. So könnte man die Funktionalitäten dadurch erweitern, dass man nur die in einem Bereich liegenden Systemknoten im Graph anzeigen lässt. 

Die nächste Idee von Herr Prof. Schubert bezieht sich auf die Wirkungskettenanalyse. In diesem Fall geht es darum, programmatisch zu ermitteln, welche Knoten von einem bestimmten Knoten aus beeinflusst werden. Es geht dabei aber nicht nur um die Knoten in unmittelbarer Nachbarschaft, sondern auch um Knoten, die durch Änderungen erst im nächsten oder übernächsten Schritt beeinflusst werden. Besonders Interessant wird es, wenn sich Wirkungskreise ergeben. Sprich wenn sich ein Knoten über mehrere andere Knoten wieder selbst beeinflusst. Diese Kreise sind systemisch gesehen sehr kritisch, da sich ein System dadurch zum Beispiel immer stärker aufschaukeln und und dadurch instabil werden kann.

Ein weiteres Problem, dass bisher noch nicht zufriedenstellend gelöst worden ist, ist die Möglichkeit, Hierarchien zu definieren. Hierarchien sollen es ermöglichen, mehrere Systemknoten zusammen zu fassen und dadurch die Übersichtlichkeit des Systems zu verbessern. Hier sieht Herr Schubert eigentlich einen Teilbaum, dessen Kindknoten in das Wurzelelement eingeklappt werden können. Das Projektteam sah eher eine einfache Gruppierung verschiedenr Knoten und deren Repräsentation durch einen logischen Platzhalterknoten. Hier gilt es weiteres Verständnis der unterschiedlichen Sichtweisen zu erlangen und über die beste Lösungsmöglichkeit anhand mehrerer Prototypen zu diskutieren.