\section{Einleitung}

Diese Studienarbeit entstand im Zuge des Langzeitprojekts KSM. 

KSM bedeutet ''Kybernetisches System Modell``. Ziel des Projektes ist es, ein Programm zu entwickeln, dass es ermöglicht komplexe Systeme, deren Knoten untereinander in starken Wechselwirkungen stehen übersichtlich zu modellieren und im Anschluss daran auch zu simulieren. Das KSM Projekt ist mittlerweile über 10 Jahre alt und wurde Jahr für Jahr im Zuge neuer Studienarbeiten weiter getrieben. Mittlerweile kann das KSM Projekt ein Programm vorweisen, dass alle Kernfunktionen bereits beinhaltet. So haben sich die Studienarbeiten der letzten Jahre eher mit der Verbesserung der Anzeige, oder der Konsolidierung des Quellcodes beschäftigt. So ist es auch Inhalt dieser Studienarbeit, umfangreiches Bugfixing und Refactoring zu betreiben, sowie eine Anzeige des Systems mit weiteren kleineren Funktionen anzureichern, um das Programm bedienbarer und übersichtlicher zu gestalten.
