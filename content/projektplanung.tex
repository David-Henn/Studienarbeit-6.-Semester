\section{Projektplanung} \label{planung}

Bei der Projektplanung sind für KSM mehrere Dinge zu berücksichtigen. 
\begin{itemize}
\item Der Zustand des Projekts
\item Die knappe Zeit
\item Die Aufgaben der Teammitglieder
\item Die Anforderungen des Projektinhabers Herr Prof. Schubert
\end{itemize}

Aufgrund des chaotischen Zustands des Quellcodes, sowie der restlichen Projektartefakte wurde in Absprache mit Herrn Prof. Schubert ein Zwei-Stufen-Plan\footnote{siehe Anhang} verabschiedet. 

Während der ersten, längeren Phase wird vom Projektteam Bugfixing und Refactoring betrieben, um das Projekt weiter zu ordnen. Um diesediese Phase optimal zu strukturieren, wird das agilen Entwicklungsmodell \emph{Scrum} eingesetzt.

Nachdem das Projekt stabilisiert wurde, soll in der zweiten Phase die Funktionalität des KSM weiterentwickelt werden. 